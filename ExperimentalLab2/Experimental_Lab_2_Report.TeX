\documentclass{article}
\usepackage{blindtext}
\usepackage{graphicx}
\usepackage[a4paper, margin=1in]{geometry}
\usepackage{amsmath}
\usepackage{subcaption}
\usepackage[export]{adjustbox}
\graphicspath{{images/}}

\title{Experimental Lab 2 Report}
\author{James Henrik Middleton (2785209M)}
\date{November 2025}

\begin{document}
\pagenumbering{gobble}
\begin{titlepage}
    \maketitle
    \center{Written in LaTeX}
\end{titlepage}

    \newpage

    \section*{Aims} %==========================================================================

    \begin{flushleft}
        The aims of this experiemnt are to invesgiate the formation of eye diagrams based from
        digital signals. Eye diagrams can help to visualize SNR, Jitter, and timing error in a digital signal. 
        We will be using a digital oscilloscope to visualize our outputs.
    \end{flushleft}

    \section*{Method} %========================================================================

    \begin{flushleft}
        We enabled infinite persistence and chose an appropriate trigger height on 
        the oscilloscope and it produced an eye diagram as shown in Figure 1.
    \end{flushleft}

    \begin{flushleft}
        We then changed the frequency of the wave to 50Hz, 100Hz, 200Hz, and 300Hz respectively to
        analyse the effect this has on the shape of the eye. The output of this is shown in Figure 2.
    \end{flushleft}

    \begin{flushleft}
        We wrote a python script to generate 2 different types of array to be input into 
        a picoscope for wave generation.
        Our script generated a wave that had 10 sample points per bit. 
        We generated 2 different types of bit sequence: on-off keying and PAM-4.
        We generated 5 different waves each with a different SNR (6, 8, 12, 18, and 24dB). 
        The output from the on-off keaying is shown in Figure 3 and the PAM-4 keying is shown in Figure 4.
    \end{flushleft}

    \begin{flushleft}
        We set the oscilloscope math function to low pass filter and gradually changed the frequency 
        from 10MHz to 40kHz and analyzed the effect on our eye diagram. 
        The output of this experiment is shown in Figure 5.
    \end{flushleft}

    \section*{Conclusion} %===================================================================================

    \begin{flushleft} %Discuss the formation of an eye diagram and the effect that jitter has on the shape
        Eye diagrams are created by overlaying multiple segments of a digital signal to create a single 
        graphical representation of the wave. They are used to measure the integrity of a digital signal.
        A "jittery" or "closed" (high signal interference) eye diagram means low integrity.
        Example (a) in Figure 1 shows a clear eye diagram whereas in example (b) the trigger is too low and therefore jitter is high.
    \end{flushleft}

    \begin{flushleft} %Discuss how different frequencies of wave changes the shape of the eye
        We tested the eye diagram with multiple different frequencies. From the results produced we can see that 
        at a low frequency the edges, or "lids" of the eye diagram will be further apart. At higher frequencies 
        they will be closer together (with the distance roughly halving proportionally to a doubling of the frequency).
        We can also see that the eye becomes gradually more rounded or starts to "close" at higher frequencies.
        This shows that higher frequencies can lead to signal loss. This could be due to interference within close peaks
        within the signal and the fact that the signal may not have time to reach peak voltage with faster oscillations.
    \end{flushleft}

    \begin{flushleft} %Discuss the effet of the signal to noise ratio on the clarity of the eye diagram with on-off keying
        We can see that a low signal to noise ratio cause extreme jitter in the eye diagram which means low integrity
        in the signal. At (6-8:1)dB the eye diagram is almost unidentifiable due to jitter. At (12:1)dB we can begin to
        see an eye diagram forming. At (18:1)dB the diagram becomes much clearer and we can begin to identify the eyelids. At
        (24:1) the signal is clear and we can identify the eyelids. This shows that at signal to noise ratios below (12:1)dB
        have too much interference to produce a legible signal with on-off keying. We can also see that the signal narrows 
        as noise decreases. This may be because the noise combines certain peaks in the signal which increases the amplitude.
    \end{flushleft}

    \begin{flushleft} %%Discuss the effet of the signal to noise ratio on the clarity of the eye diagram with PAM-4 keying
        The main difference between PAM-4 keying and on-off keying is that PAM-4 uses 4 bit levels whilst on-off uses 
        2 (binary). The way noise effects these signals is very similar. From (6-8:1)dB there is not distinguashable shape.
        From (12-18:1)dB the eye diagram begins to form. At (24:1)dB the signal becomes clear and the eyelids are identifiable.
        One of the main differences we can see is that at lower SNR the PAM-4 signal shrinks whereas the on-off signal widens.
        This may be because with high noise the 4 levels in PAM-4 interfer deleteriously which reduces signal amplitude.
    \end{flushleft}

    \begin{flushleft} %Discuss the effect of bandwith on the output of the low-pass filter
        At a 10MHz bandwidth the output is a basic square wave. There is no identifiable curvature or eye formation.
        At 1MHz the corners begin to smooth. At 40kHz the eyes become clear.
        This is beacuse as the bandwidth shrinks the high frequencies of the signal become attenuated (Their peaks are cut)
        which leads to sloping at the edges of the square wave.
    \end{flushleft}

    \begin{flushleft}
        This lab has taught us the effect that SNR, bandwidth, frequency, and trigger point has on the integrity of signals.
        We have learned how to create, read, and adjust eye diagrams.
    \end{flushleft}

    \newpage
    \section*{Results} %======================================================================

    \begin{figure}[htbp!]
    \hfill
    \center
    \begin{subfigure}{0.48\textwidth}
        \includegraphics[width=\linewidth]{scope_1.png}
        \caption{}
    \end{subfigure}
    \begin{subfigure}{0.48\textwidth}
        \includegraphics[width=\linewidth]{more_jitter_trigger_at_top.png} 
        \caption{}
    \end{subfigure}
    \caption{Eye Diagram - High and Low Trigger}
    \end{figure}
    \hfill
    \begin{figure}[htbp!]
        \begin{subfigure}{0.24\linewidth}
            \includegraphics[width=\linewidth]{50Hz.png}
        \end{subfigure}
        \begin{subfigure}{0.24\linewidth}
            \includegraphics[width=\linewidth]{100Hz.png}
        \end{subfigure}
        \begin{subfigure}{0.24\linewidth}
            \includegraphics[width=\linewidth]{200Hz.png}
        \end{subfigure}
        \begin{subfigure}{0.24\linewidth}
            \includegraphics[width=\linewidth]{300Hz.png}
        \end{subfigure}
        \caption{Frequencies [50, 100, 200, 300]Hz}
    \end{figure}

    \begin{figure}[htbp!]
        \begin{subfigure}{0.19\linewidth}
            \includegraphics[width=\linewidth]{6dB.png}
        \end{subfigure}
        \begin{subfigure}{0.19\linewidth}
            \includegraphics[width=\linewidth]{8dB.png}
        \end{subfigure}
        \begin{subfigure}{0.19\linewidth}
            \includegraphics[width=\linewidth]{12dB.png}
        \end{subfigure}
        \begin{subfigure}{0.19\linewidth}
            \includegraphics[width=\linewidth]{18dB.png}
        \end{subfigure}
        \begin{subfigure}{0.19\linewidth}
            \includegraphics[width=\linewidth]{24dB.png}
        \end{subfigure}
        \caption{On-off keying SNR ratios [6, 8, 12, 18, 24]dB}
    \end{figure}

    \begin{figure}[htbp!]
        \begin{subfigure}{0.19\linewidth}
            \includegraphics[width=\linewidth]{PAM6dB.png}
        \end{subfigure}
        \begin{subfigure}{0.19\linewidth}
            \includegraphics[width=\linewidth]{PAM8dB.png}
        \end{subfigure}
        \begin{subfigure}{0.19\linewidth}
            \includegraphics[width=\linewidth]{PAM12dB.png}
        \end{subfigure}
        \begin{subfigure}{0.19\linewidth}
            \includegraphics[width=\linewidth]{PAM18dB.png}
        \end{subfigure}
        \begin{subfigure}{0.19\linewidth}
            \includegraphics[width=\linewidth]{PAM24dB.png}
        \end{subfigure}
        \caption{PAM-4 SNR ratios [6, 8, 12, 18, 24]dB}
    \end{figure}

    \begin{figure}[htbp!]
        \begin{subfigure}{0.33\linewidth}
            \includegraphics[width=\linewidth]{10MHzBandwidth.png}
        \end{subfigure}
        \begin{subfigure}{0.33\linewidth}
            \includegraphics[width=\linewidth]{1MHzBandwidth.png}
        \end{subfigure}
        \begin{subfigure}{0.33\linewidth}
            \includegraphics[width=\linewidth]{40kHzBandwidth.png}
        \end{subfigure}
        \caption{Low-Pass Filter Bandwidth [10MHz, 1MHz, 40kHz]}
    \end{figure}

\end{document}
